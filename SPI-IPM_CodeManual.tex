% Options for packages loaded elsewhere
\PassOptionsToPackage{unicode}{hyperref}
\PassOptionsToPackage{hyphens}{url}
%
\documentclass[
]{book}
\title{SPI-IPM Code Manual}
\author{Chloé R. Nater}
\date{2022-04-21}

\usepackage{amsmath,amssymb}
\usepackage{lmodern}
\usepackage{iftex}
\ifPDFTeX
  \usepackage[T1]{fontenc}
  \usepackage[utf8]{inputenc}
  \usepackage{textcomp} % provide euro and other symbols
\else % if luatex or xetex
  \usepackage{unicode-math}
  \defaultfontfeatures{Scale=MatchLowercase}
  \defaultfontfeatures[\rmfamily]{Ligatures=TeX,Scale=1}
\fi
% Use upquote if available, for straight quotes in verbatim environments
\IfFileExists{upquote.sty}{\usepackage{upquote}}{}
\IfFileExists{microtype.sty}{% use microtype if available
  \usepackage[]{microtype}
  \UseMicrotypeSet[protrusion]{basicmath} % disable protrusion for tt fonts
}{}
\makeatletter
\@ifundefined{KOMAClassName}{% if non-KOMA class
  \IfFileExists{parskip.sty}{%
    \usepackage{parskip}
  }{% else
    \setlength{\parindent}{0pt}
    \setlength{\parskip}{6pt plus 2pt minus 1pt}}
}{% if KOMA class
  \KOMAoptions{parskip=half}}
\makeatother
\usepackage{xcolor}
\IfFileExists{xurl.sty}{\usepackage{xurl}}{} % add URL line breaks if available
\IfFileExists{bookmark.sty}{\usepackage{bookmark}}{\usepackage{hyperref}}
\hypersetup{
  pdftitle={SPI-IPM Code Manual},
  pdfauthor={Chloé R. Nater},
  hidelinks,
  pdfcreator={LaTeX via pandoc}}
\urlstyle{same} % disable monospaced font for URLs
\usepackage{color}
\usepackage{fancyvrb}
\newcommand{\VerbBar}{|}
\newcommand{\VERB}{\Verb[commandchars=\\\{\}]}
\DefineVerbatimEnvironment{Highlighting}{Verbatim}{commandchars=\\\{\}}
% Add ',fontsize=\small' for more characters per line
\usepackage{framed}
\definecolor{shadecolor}{RGB}{248,248,248}
\newenvironment{Shaded}{\begin{snugshade}}{\end{snugshade}}
\newcommand{\AlertTok}[1]{\textcolor[rgb]{0.94,0.16,0.16}{#1}}
\newcommand{\AnnotationTok}[1]{\textcolor[rgb]{0.56,0.35,0.01}{\textbf{\textit{#1}}}}
\newcommand{\AttributeTok}[1]{\textcolor[rgb]{0.77,0.63,0.00}{#1}}
\newcommand{\BaseNTok}[1]{\textcolor[rgb]{0.00,0.00,0.81}{#1}}
\newcommand{\BuiltInTok}[1]{#1}
\newcommand{\CharTok}[1]{\textcolor[rgb]{0.31,0.60,0.02}{#1}}
\newcommand{\CommentTok}[1]{\textcolor[rgb]{0.56,0.35,0.01}{\textit{#1}}}
\newcommand{\CommentVarTok}[1]{\textcolor[rgb]{0.56,0.35,0.01}{\textbf{\textit{#1}}}}
\newcommand{\ConstantTok}[1]{\textcolor[rgb]{0.00,0.00,0.00}{#1}}
\newcommand{\ControlFlowTok}[1]{\textcolor[rgb]{0.13,0.29,0.53}{\textbf{#1}}}
\newcommand{\DataTypeTok}[1]{\textcolor[rgb]{0.13,0.29,0.53}{#1}}
\newcommand{\DecValTok}[1]{\textcolor[rgb]{0.00,0.00,0.81}{#1}}
\newcommand{\DocumentationTok}[1]{\textcolor[rgb]{0.56,0.35,0.01}{\textbf{\textit{#1}}}}
\newcommand{\ErrorTok}[1]{\textcolor[rgb]{0.64,0.00,0.00}{\textbf{#1}}}
\newcommand{\ExtensionTok}[1]{#1}
\newcommand{\FloatTok}[1]{\textcolor[rgb]{0.00,0.00,0.81}{#1}}
\newcommand{\FunctionTok}[1]{\textcolor[rgb]{0.00,0.00,0.00}{#1}}
\newcommand{\ImportTok}[1]{#1}
\newcommand{\InformationTok}[1]{\textcolor[rgb]{0.56,0.35,0.01}{\textbf{\textit{#1}}}}
\newcommand{\KeywordTok}[1]{\textcolor[rgb]{0.13,0.29,0.53}{\textbf{#1}}}
\newcommand{\NormalTok}[1]{#1}
\newcommand{\OperatorTok}[1]{\textcolor[rgb]{0.81,0.36,0.00}{\textbf{#1}}}
\newcommand{\OtherTok}[1]{\textcolor[rgb]{0.56,0.35,0.01}{#1}}
\newcommand{\PreprocessorTok}[1]{\textcolor[rgb]{0.56,0.35,0.01}{\textit{#1}}}
\newcommand{\RegionMarkerTok}[1]{#1}
\newcommand{\SpecialCharTok}[1]{\textcolor[rgb]{0.00,0.00,0.00}{#1}}
\newcommand{\SpecialStringTok}[1]{\textcolor[rgb]{0.31,0.60,0.02}{#1}}
\newcommand{\StringTok}[1]{\textcolor[rgb]{0.31,0.60,0.02}{#1}}
\newcommand{\VariableTok}[1]{\textcolor[rgb]{0.00,0.00,0.00}{#1}}
\newcommand{\VerbatimStringTok}[1]{\textcolor[rgb]{0.31,0.60,0.02}{#1}}
\newcommand{\WarningTok}[1]{\textcolor[rgb]{0.56,0.35,0.01}{\textbf{\textit{#1}}}}
\usepackage{longtable,booktabs,array}
\usepackage{calc} % for calculating minipage widths
% Correct order of tables after \paragraph or \subparagraph
\usepackage{etoolbox}
\makeatletter
\patchcmd\longtable{\par}{\if@noskipsec\mbox{}\fi\par}{}{}
\makeatother
% Allow footnotes in longtable head/foot
\IfFileExists{footnotehyper.sty}{\usepackage{footnotehyper}}{\usepackage{footnote}}
\makesavenoteenv{longtable}
\usepackage{graphicx}
\makeatletter
\def\maxwidth{\ifdim\Gin@nat@width>\linewidth\linewidth\else\Gin@nat@width\fi}
\def\maxheight{\ifdim\Gin@nat@height>\textheight\textheight\else\Gin@nat@height\fi}
\makeatother
% Scale images if necessary, so that they will not overflow the page
% margins by default, and it is still possible to overwrite the defaults
% using explicit options in \includegraphics[width, height, ...]{}
\setkeys{Gin}{width=\maxwidth,height=\maxheight,keepaspectratio}
% Set default figure placement to htbp
\makeatletter
\def\fps@figure{htbp}
\makeatother
\setlength{\emergencystretch}{3em} % prevent overfull lines
\providecommand{\tightlist}{%
  \setlength{\itemsep}{0pt}\setlength{\parskip}{0pt}}
\setcounter{secnumdepth}{5}
\usepackage{booktabs}
\usepackage{amsthm}
\makeatletter
\def\thm@space@setup{%
  \thm@preskip=8pt plus 2pt minus 4pt
  \thm@postskip=\thm@preskip
}
\makeatother
\ifLuaTeX
  \usepackage{selnolig}  % disable illegal ligatures
\fi
\usepackage[]{natbib}
\bibliographystyle{apalike}

\begin{document}
\maketitle

{
\setcounter{tocdepth}{1}
\tableofcontents
}
\hypertarget{about-this-manual}{%
\chapter*{About this manual}\label{about-this-manual}}
\addcontentsline{toc}{chapter}{About this manual}

This manual is intended as a user guide for the SPI-IPM repository (\url{https://github.com/SPI-Birds/SPI-IPM}). The repository contains an entire toolbox of code to implement, run, and analyse integrated population models (IPMs) using data hosted by the SPI-Birds Network and Database (\url{https://spibirds.org}).While code is designed to work directly with the SPI-Birds standard data format and the population model is tailored to hole-nesting bird species, both code and workflow can serve as a basis for extensions and applications to a wide range of other species and data types.

The increasing adoption of standard formats for ecological data opens new possibilities for (comparative) analyses at much larger spatial scales. Accessible, transparent, and reproducible tools and workflows are central to implementing such analyses. The idea behind SPI-IPM is to provide such tools and workflows for integrated analyses of population dynamics using Bayesian IPMs.

IPMs have rapidly gained popularity over the past decade due to a range of advantages including efficient use of available data, thorough inclusion and propagation of uncertainty, and estimation of parameters for which no explicit data can be collected. To learn more about IPMs and their applications in general, see \citet{plard2019} and \citet{schaub2021}. The use of IPMs with SPI-Birds data for identifying and quantifying the drivers of population dynamics (of pied flycatchers) is illustrated in the paper that gave rise to the SPI-IPM repository.

This manual is organised into eight chapters that cover the entire workflow presented in Figure \ref{fig:WorkflowDiag}:

\begin{figure}

{\centering \includegraphics[width=1\linewidth]{Figures/SPI-IPM_Workflow} 

}

\caption{Schematic representation of the SPI-IPM workflow.}\label{fig:WorkflowDiag}
\end{figure}

Chapter 1 describes how data obtained in the \href{https://github.com/SPI-Birds/documentation/tree/master/standard_protocol}{SPI-Birds standard format} is reformatted for use in an IPM. Chapters 2 \& 3 explain the structure and assumptions of the IPM, and chapter 4 details its implementation using the R package \texttt{nimble} \citep{devalpine2017}. Chapter 5 introduces ways for assessing model performance and fit and chapter 6 contains notes about visualizing results in different ways. A range of useful follow-up analyses (incl.~tests for time trends and density dependence, perturbation analyses, etc.) are described in Chapter 7. The final chapter (8) is more of a teaser trailer than anything else, presenting a few ideas for future extensions and additional applications.

Relevant code for each chapter is deposited in subfolders in the \href{https://github.com/SPI-Birds/SPI-IPM/tree/main/SPI-IPM_Code}{code repository} that start with the same number as the manual chapter. The repository README contains more details about where to find what.

At present, this manual is incomplete in two ways. First, there are chapters for which content is still missing (marked with ``TBA''). This content will be added throughout 2022. Second, and more importantly: neither the code repository nor the manual are exhaustive. They are meant to help others use and adapt SPI-IPM for their own studies and will hopefully keep growing as more people add their own extensions and new applications to the repository.

\hypertarget{how-to-cite}{%
\subsection*{How to cite}\label{how-to-cite}}
\addcontentsline{toc}{subsection}{How to cite}

TBA

\hypertarget{DataPrep}{%
\chapter{Preparing SPI-Birds data for Bayesian analysis}\label{DataPrep}}

TBA

\hypertarget{nest-count-data}{%
\section{Nest count data}\label{nest-count-data}}

TBA

\hypertarget{clutch-size-data}{%
\section{Clutch size data}\label{clutch-size-data}}

TBA

\hypertarget{nest-level}{%
\subsection{Nest level}\label{nest-level}}

TBA

\hypertarget{population-level}{%
\subsection{Population level}\label{population-level}}

TBA

\hypertarget{fledgling-count-data}{%
\section{Fledgling count data}\label{fledgling-count-data}}

TBA

\hypertarget{nest-level-1}{%
\subsection{Nest level}\label{nest-level-1}}

TBA

\hypertarget{population-level-1}{%
\subsection{Population level}\label{population-level-1}}

TBA

\hypertarget{mark-recapture-data}{%
\section{Mark-recapture data}\label{mark-recapture-data}}

TBA

\hypertarget{individual-capture-histories}{%
\subsection{Individual capture histories}\label{individual-capture-histories}}

TBA

\hypertarget{m-array}{%
\subsection{M-array}\label{m-array}}

TBA

\hypertarget{immigrant-count-data}{%
\section{Immigrant count data}\label{immigrant-count-data}}

TBA

\hypertarget{auxiliary-data-on-the-sampling-process}{%
\section{Auxiliary data on the sampling process}\label{auxiliary-data-on-the-sampling-process}}

TBA

\hypertarget{nest-survey-sampling-effort}{%
\subsection{Nest survey sampling effort}\label{nest-survey-sampling-effort}}

TBA

\hypertarget{capture-probability-proxies}{%
\subsection{Capture probability proxies}\label{capture-probability-proxies}}

TBA

\hypertarget{organisation-for-analysis-with-nimble}{%
\section{Organisation for analysis with NIMBLE}\label{organisation-for-analysis-with-nimble}}

TBA

\hypertarget{IPMCon}{%
\chapter{IPM Construction}\label{IPMCon}}

\hypertarget{open-population-model-with-2-age-classes}{%
\section{Open population model with 2 age classes}\label{open-population-model-with-2-age-classes}}

\hypertarget{model-description}{%
\subsection{Model description}\label{model-description}}

Population dynamics are represented using a female-based age-structured open
population model with a pre-breeding census in spring. At census, females are
divided into two age classes: ``yearlings'' (1-year old birds hatched during the
breeding season of the previous year) and ``adults'' (birds older than one year).
The motivation underlying this distinction is that reproductive output often
differs for these two age classes in passerine birds.
The dynamics of the female segment of the population over the time-interval from
census in year \(t\) to census in year \(t+1\) can be described with classic matrix
notation \citep{Caswell2001} as:

\hfill\break
\[N_{tot,t+1} = \begin{bmatrix} N_{Y,t+1} \\ N_{A,t+1} \end{bmatrix} =
  \begin{bmatrix}
0.5F_{Y,t}sJ_t & 0.5F_{A,t}sJ_t \\
sA_t & sA_t
\end{bmatrix}\begin{bmatrix} N_{Y,t} \\ N_{A,t} \end{bmatrix} +
  \begin{bmatrix} Imm_{Y,t+1} \\ Imm_{A,t+1} \end{bmatrix}\]\\

\(N_{tot,t+1}\) represents the total number of yearling and adult females in the
population upon arrival in the breeding areas in year t. The total female
population size, \(N_{tot,t+1}\), is the sum of the numbers of yearling and adult
females in the population in year \(t+1\) (\(N_{Y,t+1}\) and \(N_{A,t+1}\),
respectively) and consists of local survivors and recruits from the previous
breeding season, as well as immigrant yearling (\(Imm_{Y,t+1}\)) and adult
(\(Imm_{A,t+1}\)) females.\\

\(F_{a,t}\) represents the expected number of fledglings produced by age class \(a\)
females during the breeding season in year \(t\) and is the product of several
vital rates. First, females in age class a may breed in a nestbox with
probability \(pB_{a,t}\) upon arrival to breeding areas in year \(t\). Each breeding
female may then lay a clutch containing a certain number of eggs (expected
number = \(CS_{a,t}\)), and each egg within the clutch may hatch and survive to
fledging. The probability of an egg hatching and surviving to fledging is
divided into an age-independent probability of nest success (\(pNS_t\),
probability of complete clutch failure = \(1-pNS_t\)) and a survival probability
of every egg/chick to fledging provided that the nest has not failed entirely
(\(sN_{a,t}\), with \(a\) = age of the mother). Consequently, the expected number of
fledglings produced by age class \(a\) females in year t is defined as:

\begin{equation}
F_{a,t}= pB_{a,t}\times CS_{a,t}\times pNS_{t}\times sN_{a,t}
\end{equation}

Fledglings that survive to the next breeding season and remain within the
population (probability = \(sJ_t\)) contribute to next year's yearling class
(\(N_{Y,t+1}\)). Yearlings and adults that survive to the next breeding season and
remain within the population (probability = \(sA_t\)) become part of next year's
adult age class (\(N_{A,t+1}\)).

\hypertarget{code-implementation-including-demographic-stochasticity}{%
\subsection{Code implementation including demographic stochasticity}\label{code-implementation-including-demographic-stochasticity}}

Population process models within IPMs are typically implemented as stochastic
models that account for randomness in the outcomes of demographic processes at
the individual level \citep[``demographic stochasticity'',][]{Caswell2001, kery2011}.
The model described here is no different, meaning that the numbers of breeders,
fledglings, and survivors are treated as binomial and Poisson random variables.

Reproduction is modelled via two sets of random variables: a binomial random
variable representing the number of breeders in age class \(a\) in year \(t\),
\(B_{a,t}\) and a Poisson random variable representing the
number of fledlings produced by breeders of age class \(a\) in year \(t\),
\(Juv_{a,t}\) . The implementation in the BUGS language used in the
SPI-IPM code (\href{https://github.com/SPI-Birds/SPI-IPM/blob/main/SPI-IPM_Code/02-04_IPM_Setup\&Run/IPMSetup.R}{\texttt{IPMSetup.R}}, lines 231-241) looks like:

\begin{Shaded}
\begin{Highlighting}[]
\ControlFlowTok{for}\NormalTok{ (t }\ControlFlowTok{in} \DecValTok{1}\SpecialCharTok{:}\NormalTok{Tmax)\{}
  \ControlFlowTok{for}\NormalTok{(a }\ControlFlowTok{in} \DecValTok{1}\SpecialCharTok{:}\NormalTok{A)\{}
    
    \DocumentationTok{\#\# 1) Breeding decision}
\NormalTok{    B[a,t] }\SpecialCharTok{\textasciitilde{}} \FunctionTok{dbin}\NormalTok{(pB[a,t], N[a,t])}
    
    \DocumentationTok{\#\# 2) Offspring production}
\NormalTok{    Juv[a,t] }\SpecialCharTok{\textasciitilde{}} \FunctionTok{dpois}\NormalTok{(B[a,t]}\SpecialCharTok{*}\NormalTok{CS[a,t]}\SpecialCharTok{*}\NormalTok{pNS[t]}\SpecialCharTok{*}\NormalTok{sN[a,t]}\SpecialCharTok{*}\FloatTok{0.5}\NormalTok{)}
\NormalTok{  \}}
\NormalTok{\}}
\end{Highlighting}
\end{Shaded}

In the code, the age indeces \(a=1\) and \(a=A=2\) correspond to yearlings and
adults, respectively.

Analogous to breeders, the numbers of local survivors -- both fledglings surviving their
first year and becoming yearlings, and yearlings and adults surviving to the
next year -- are implemented as binomial random variables (\href{https://github.com/SPI-Birds/SPI-IPM/blob/main/SPI-IPM_Code/02-04_IPM_Setup\&Run/IPMSetup.R}{\texttt{IPMSetup.R}}, lines 243-255):

\begin{Shaded}
\begin{Highlighting}[]
\ControlFlowTok{for}\NormalTok{ (t }\ControlFlowTok{in} \DecValTok{1}\SpecialCharTok{:}\NormalTok{(Tmax}\DecValTok{{-}1}\NormalTok{))\{}
  
  \DocumentationTok{\#\# 3) Annual survival of local birds}
  \CommentTok{\# Juveniles {-}\textgreater{} Yearlings}
\NormalTok{  localN[}\DecValTok{1}\NormalTok{,t}\SpecialCharTok{+}\DecValTok{1}\NormalTok{] }\SpecialCharTok{\textasciitilde{}} \FunctionTok{dbin}\NormalTok{(sJ[t], }\FunctionTok{sum}\NormalTok{(Juv[}\DecValTok{1}\SpecialCharTok{:}\NormalTok{A,t]))}
  \CommentTok{\# Yearlings/Adults {-}\textgreater{} adults}
\NormalTok{  localN[}\DecValTok{2}\NormalTok{,t}\SpecialCharTok{+}\DecValTok{1}\NormalTok{] }\SpecialCharTok{\textasciitilde{}} \FunctionTok{dbin}\NormalTok{(sA[t], }\FunctionTok{sum}\NormalTok{(N[}\DecValTok{1}\SpecialCharTok{:}\NormalTok{A,t]))}
  
  \DocumentationTok{\#\# 4) Immigration}
  \ControlFlowTok{for}\NormalTok{(a }\ControlFlowTok{in} \DecValTok{1}\SpecialCharTok{:}\NormalTok{A)\{}
\NormalTok{    N[a,t}\SpecialCharTok{+}\DecValTok{1}\NormalTok{] }\OtherTok{\textless{}{-}}\NormalTok{ localN[a,t}\SpecialCharTok{+}\DecValTok{1}\NormalTok{] }\SpecialCharTok{+}\NormalTok{ Imm[a,t}\SpecialCharTok{+}\DecValTok{1}\NormalTok{]}
\NormalTok{  \}}
\NormalTok{\}}
\end{Highlighting}
\end{Shaded}

Immigrant numbers are also treated as outcomes of stochastic processes, and
these are detailed in \protect\hyperlink{ux5cux23ux5cux23ux5cux2520Immigrantux5cux2520countux5cux2520dataux5cux2520likelihood}{2.2.5 Immigrant count data likelihood}.

\hypertarget{data-likelihoods}{%
\section{Data likelihoods}\label{data-likelihoods}}

IPMs obtain information on the population model's parameters (population sizes
and vital rates) from several different data sets. Information in each data set
is channeled into model parameters via one or multiple data likelihoods.\\
The SPI-IPM contains five data modules consisting of a total of eight data
likelihoods: nest count data (one likelihood), clutch size data (two likelihoods),
fledgling count data (three likelihoods), mark-recapture data (one likelihood),
and immigrant count data (one likelihood).
The likelihoods contained in each data module are described in detail in the
following sub-chapters. The underlying data sets are introduced in
\protect\hyperlink{DataPrep}{Chapter 1} of the manual.

\hypertarget{nest-count-data-likelihood}{%
\subsection{Nest count data likelihood}\label{nest-count-data-likelihood}}

The population model defines the true size of the female segment of the breeding
population in any year \(t\) via the year- and age-specific number of breeding
females:

\begin{equation}
B_{a,t}  \sim Binomial(N_{a,t}, pB_{a,t})
\end{equation}

The total size of the female breeding population (the sum of \(B_{a,t}\) over all
age classes) in year \(t\) is expected to correspond closely with the observed
number of first clutches laid in any year (\(NestCount_t\)).
The breeding population count thus contains information about both breeding
probability (\(pB_{a,t}\)) and population size (\(N_{a,t}\)) and its likelihood can
be defined as:

\begin{equation}
NestCount_t  \sim Poisson((B_{Y,t} + B_{A,t}) \times NS\_Data_t)
\end{equation}

Here, the year-specific variable \(NS\_Data_t\) is a correction factor introduced
for dealing with years in which no nest survey data was collected
(\(NestCount_t = 0\) due to lack of sampling). \(NS\_Data_t\) is set to 0 for years
without data collection, and to 1 in all other years.

The likelihood is written almost identically in the code, the only difference
being the use the \texttt{sum()} function over age classes \(1\) to \(A=2\) when
specifying the Poisson distribution (this provides more flexibility for
extending the model to more than two age classes):

\begin{Shaded}
\begin{Highlighting}[]
\ControlFlowTok{for}\NormalTok{(t }\ControlFlowTok{in} \DecValTok{1}\SpecialCharTok{:}\NormalTok{Tmax)\{}
\NormalTok{  NestCount[t] }\SpecialCharTok{\textasciitilde{}} \FunctionTok{dpois}\NormalTok{(}\FunctionTok{sum}\NormalTok{(B[}\DecValTok{1}\SpecialCharTok{:}\NormalTok{A,t])}\SpecialCharTok{*}\NormalTok{NS\_Data[t])}
\NormalTok{\}}
\end{Highlighting}
\end{Shaded}

\hypertarget{clutch-size-data-likelihoods}{%
\subsection{Clutch size data likelihoods}\label{clutch-size-data-likelihoods}}

The counting of incubated eggs provides information about both individual-level
clutch size (\(CS_{a,t}\)) and reproductive output at the population level.
Consequently, two separate likelihoods can be specified for within the clutch
size data module.

Each individual clutch size observation can be treated as the outcome of a
Poisson process with an expected value of \(CS_{a,t}\) (where \(a\) is the age of
the female that laid the clutch, and \(t\) the year in which the clutch was laid).
In the code, the likelihood is formulated for each clutch size observation \(x\)
(out of a total of \texttt{CS\_X} observations) and includes nested indexing of the
expectation using data on female age (\texttt{CS\_FAge}) and year (\texttt{CS\_year}):

\begin{Shaded}
\begin{Highlighting}[]
\ControlFlowTok{for}\NormalTok{(x }\ControlFlowTok{in} \DecValTok{1}\SpecialCharTok{:}\NormalTok{CS\_X)\{}
\NormalTok{    ClutchSize[x] }\SpecialCharTok{\textasciitilde{}} \FunctionTok{dpois}\NormalTok{(CS[CS\_FAge[x], CS\_year[x]])}
\NormalTok{\}}
\end{Highlighting}
\end{Shaded}

Since both year and female age need to be part of the provided data, the
individual-level clutch size likelihood can only be used with complete
observations, i.e.~clutches for which both number of eggs and age of the mother
are known (year should always be known).

For the population-level likelihood, on the other hand, data on clutch sizes can
be included irrespective of whether the age of the mother is known. The total
number of eggs counted in all nests laid in year \(t\) can be described as:

\begin{equation}
EggNoTot_t \sim Poisson(sum(B_{Y:A,t} \times CS_{Y:A,t}) \times p_t^{EggNo})
\end{equation}

Since the product \(B_{a,t} \times CS_{a,t}\) corresponds to all eggs laid in all nests/nestboxes within the study site, another correction factor (\(p_t^{EggNo}\)) is needed to account for the fact that eggs may not have been counted in all surveyed nests with breeding activity in each year. \(p_t^{EggNo}\) (\texttt{EggNoSP{[}t{]}} in code) thus contains information of the year-specific proportionof surveyed nests for which the numbers of eggs were counted.

In code, the likelihood for the number of eggs at the population is written as:

\begin{Shaded}
\begin{Highlighting}[]
\ControlFlowTok{for}\NormalTok{(t }\ControlFlowTok{in} \DecValTok{1}\SpecialCharTok{:}\NormalTok{Tmax)\{}

    \CommentTok{\# Expected "true" egg number (by mother age class)}
\NormalTok{    EggNo.ex[}\DecValTok{1}\SpecialCharTok{:}\NormalTok{A,t] }\OtherTok{\textless{}{-}}\NormalTok{ B[}\DecValTok{1}\SpecialCharTok{:}\NormalTok{A,t]}\SpecialCharTok{*}\NormalTok{CS[}\DecValTok{1}\SpecialCharTok{:}\NormalTok{A,t]}

    \CommentTok{\# Observed egg number (corrected by data availaility)}
\NormalTok{    EggNoTot[t] }\SpecialCharTok{\textasciitilde{}} \FunctionTok{dpois}\NormalTok{(}\FunctionTok{sum}\NormalTok{(EggNo.ex[}\DecValTok{1}\SpecialCharTok{:}\NormalTok{A,t])}\SpecialCharTok{*}\NormalTok{EggNoSP[t])}

\NormalTok{\}}
\end{Highlighting}
\end{Shaded}

\hypertarget{fledgling-count-data-likelihoods}{%
\subsection{Fledgling count data likelihoods}\label{fledgling-count-data-likelihoods}}

Analogous to observations of clutch size, counts of fledglings contain
information on reproduction at both individual and population level.
Distributions of number of fledglings produced from a clutch are often 0-inflated
because incidents of harsh weather, predation, adandonment, etc. may result in
complete loss of entire clutches. To account for this, we split data on fledgling
numbers and formulated separate likelihoods for the survival of the clutch as
a whole (probability of a nest not failing completely, \(pNS_t\)) and for each
chick subsequently surviving to fledgling (probability \(sN_{a,t}\)).\\
Whether or not a clutch succeeded (i.e.~at least one chick survived to fledging,
\texttt{anyFledged}) was coded using 1 (success) and 0 (failure) and modelled as the
outcome of a Bernoulli process with a year-dependent success probability \(pNS_t\):

\begin{Shaded}
\begin{Highlighting}[]
\ControlFlowTok{for}\NormalTok{(x }\ControlFlowTok{in} \DecValTok{1}\SpecialCharTok{:}\NormalTok{F\_X)\{}
\NormalTok{  anyFledged[x] }\SpecialCharTok{\textasciitilde{}} \FunctionTok{dbern}\NormalTok{(pNS[F\_year[x]])}
\NormalTok{\}}
\end{Highlighting}
\end{Shaded}

For the subset of successful clutches, the number of fledglings produced
(\texttt{NoFledged}) was modelled as a binomial random variable with each egg laid in
the clutch (\texttt{NoLaid}) having a probability of \(sN_{a,t}\) (where \(a\) = age of the
mother) to survive and fledge:

\begin{Shaded}
\begin{Highlighting}[]
\ControlFlowTok{for}\NormalTok{(x }\ControlFlowTok{in} \DecValTok{1}\SpecialCharTok{:}\NormalTok{NoF\_X)\{}
\NormalTok{    NoFledged[x] }\SpecialCharTok{\textasciitilde{}} \FunctionTok{dbin}\NormalTok{(sN[NoF\_FAge[x], NoF\_year[x]], NoLaid[x])}
\NormalTok{\}}
\end{Highlighting}
\end{Shaded}

At the population level, both processes (nest success and survival to fledging
conditional on nest success) were combined into a single Poisson
likelihood describing the total number of fledglings counted in the population
in a given year \(t\) as the product of the number of breeding females, clutch
size, nest success, and survival to fledging:

\begin{equation}
FledgedTot_t  \sim Poisson(sum(B_{Y:A,t}\times CS_{Y:A,t}\times pNS_t\times sN_{Y:A,t}) \times p_t^{Fledged})
\end{equation}

As in the clutch size data module (\protect\hyperlink{ux5cux23ux5cux23ux5cux2520Clutchux5cux2520sizeux5cux2520dataux5cux2520likelihoods}{Chapter 2.2.2}),
a correction factor quantifying the proportion of nests for which data is
availeble (\(p_t^{Fledged}\), \texttt{FledgedSP{[}t{]}} in code) is used to account for
missing records of fledgling numbers.

In the code, calculation of the expected number of fledglings is based on the expected number of eggs as calculated in the clutch size data module
(\protect\hyperlink{ux5cux23ux5cux23ux5cux2520Clutchux5cux2520sizeux5cux2520dataux5cux2520likelihoods}{Chapter 2.2.2}):

\begin{Shaded}
\begin{Highlighting}[]
\ControlFlowTok{for}\NormalTok{(t }\ControlFlowTok{in} \DecValTok{1}\SpecialCharTok{:}\NormalTok{Tmax)\{}

    \CommentTok{\# Expected "true" fledgling number (by mother age class)}
\NormalTok{    Fledged.ex[}\DecValTok{1}\SpecialCharTok{:}\NormalTok{A,t] }\OtherTok{\textless{}{-}}\NormalTok{ EggNo.ex[}\DecValTok{1}\SpecialCharTok{:}\NormalTok{A,t]}\SpecialCharTok{*}\NormalTok{pNS[t]}\SpecialCharTok{*}\NormalTok{sN[}\DecValTok{1}\SpecialCharTok{:}\NormalTok{A,t]}

    \CommentTok{\# Observed egg number (corrected by data availaility)}
\NormalTok{    FledgedTot[t] }\SpecialCharTok{\textasciitilde{}} \FunctionTok{dpois}\NormalTok{(}\FunctionTok{sum}\NormalTok{(Fledged.ex[}\DecValTok{1}\SpecialCharTok{:}\NormalTok{A,t])}\SpecialCharTok{*}\NormalTok{FledgedSP[t])}
\NormalTok{\}}
\end{Highlighting}
\end{Shaded}

\hypertarget{mark-recapture-data-likelihood}{%
\subsection{Mark-recapture data likelihood}\label{mark-recapture-data-likelihood}}

In the SPI-IPM, capture histories of marked birds are analysed using an age-specific Cormack-Jolly-Seber (CJS) model \citep{cormack1964, jolly1965, seber1965}, which allows estimation of parameters associated with both annual survival and the (re-)capturing process.\\
The survival parameters \(sJ_t\) and \(sA_t\) describe the probability of surviving from one breeding season to the next for fledglings/juveniles and adult females, respectively. Since field-based sexing fledglings is not possible in the hatch year, first-year survival in the CJS likelihood was set to \(0.5sJ_t\), representing the probability that a fledgling is female and survives (recaptures of adult males were removed from the capture histories).\\
The time- and age-dependent recapture parameters \(p_{Age,t}^{Recap}\) describe the probability that an age class \(a\) individual is recaptured and identified during the breeding season of year \(t\) given that it survived the time interval \(t-1\) to \(t\). For many of the bird populations included in SPI-Birds, (re-)captures of birds are carried out at the nestboxes during the breeding season only. Consequently, two conditions need to be met for a (re-)capture besides the individual being alive: 1) the individual breeds in a nestbox (as non-breeders and birds breeding in natural cavities are not captured) and 2) the individual is actually captured and marked/identified while breeding in a nestbox. The recapture probability is therefore the product of the probability of breeding in a nestbox (\(pB_{Age,t}\)) and the probability of capture and identification given breeding in a nestbox (\(p_t^{CapB}\)):
\begin{equation}
  p_{Age,t}^{Recap}=pB_{Age,t}\times p_t^{CapB}
\end{equation}
The parameters \(pB_{Age,t}\) and \(p_t^{CapB}\) are confounded, and auxiliary data on one of them is required to separately estimate them. Here, raw data on the annual proportions of surveyed nests for which the identity of the breeding female was recorded (as a result of ringing or recapture) is used to approximated \(p_t^{CapB}\), allowing the CJS model to estimate age- and time-specific breeding probabilities (\(pB_{Age,t}\)).\\
CJS models in Bayesian frameworks have traditionally been implemented either as latent states models with Bernouilli likelihoods or as multinomial likelihoods for data re-formatted as ``m-arrays'' \citep{gimenez2007, kery2011}. The former typically results in long runtimes and high computational costs, while the improved efficiency of the latter is tied to an implementation that suffers from being neither particularly intuitive nor user-friendly. To avoid these pitfalls, the CJS model in the SPI-IPM is implemented using a much more efficient marginalized likelihood (that integrates over all latent states) that can be applied to unique (not individual) capture histories only \citep[following][]{turek2016}:

\begin{Shaded}
\begin{Highlighting}[]
\DocumentationTok{\#\# Likelihood with custom distribution}
\ControlFlowTok{for}\NormalTok{ (i }\ControlFlowTok{in} \DecValTok{1}\SpecialCharTok{:}\NormalTok{n.CH)\{}

\NormalTok{  y.sum[i, first.sum[i]}\SpecialCharTok{:}\NormalTok{last.sum[i]] }\SpecialCharTok{\textasciitilde{}} \FunctionTok{dCJS\_vv\_sum}\NormalTok{(}
      \AttributeTok{probSurvive =}\NormalTok{ phi.CH[i, first.sum[i]}\SpecialCharTok{:}\NormalTok{last.sum[i]],}
      \AttributeTok{probCapture =}\NormalTok{ p.CH[i, first.sum[i]}\SpecialCharTok{:}\NormalTok{last.sum[i]],}
      \AttributeTok{len =}\NormalTok{ last.sum[i]}\SpecialCharTok{{-}}\NormalTok{first.sum[i]}\SpecialCharTok{+}\DecValTok{1}\NormalTok{,}
      \AttributeTok{mult =}\NormalTok{ CHs.count[i])}
\NormalTok{\}}
\end{Highlighting}
\end{Shaded}

In practice, this involves the use of a custom distribution \texttt{dCJS\_vv\_sum}, and more information on the likelihood and its implementation using NIMBLE is provided in \protect\hyperlink{ux5cux23ux5cux2520Efficientux5cux2520implementationux5cux2520usingux5cux2520NIMBLE}{Chapter 4.1 (Efficient implementation using NIMBLE)}.

\hypertarget{immigrant-count-data-likelihood}{%
\subsection{Immigrant count data likelihood}\label{immigrant-count-data-likelihood}}

In many nestbox studies, chick ringing is an exhaustive effort and it is not unreasonable to assume that all chicks hatched in nestboxes within a study population are ringed before fledging. Similarly, assuming that females captured in nestboxes are breeding (or attempting to breed) is often not far from the truth. If those two assumptions are largely met, then the annual numbers of newly ringed adult females (\(ImmNoObs_t\)) can provide information about immigration or -- more specifically -- about the latent number of newly immigrated females breeding in nestboxes each year (\(ImmNoTot_t\)). Typically, age will be unknown for immigrant females (unless they were marked as chicks elsewhere) and hence there is no age index on \(ImmNoObs_t\) and \(ImmNoTot_t\). In \texttt{SPI-IPM}, we use a Poisson observation model to describe the relationship between observed and true numbers of breeding immigrant females:
\begin{equation}
  ImmNoObs_t \sim Poisson(ImmNoTot_t \times p_t^{ImmDetect})
\end{equation}
The expected value represents the true number of female breeding immigrants in year \(t\) (\(ImmNoTot_t\)) corrected by an annual probability of detecting (= ringing) such individuals (\(p_t^{ImmDetect}\)). In most nestbox studies, ringing and recapturing of breeding birds are conducted in tandem as part of the same protocol, and we can consequently approximate adult ringing probability using recapture probability (as used in the likelihood for capture histories): \(p_t^{ImmDetect} = p_t^{CapB}\)). In a few exceptional years in some datasets, however, adult birds may have been ringed, but no recaptures recorded (i.e.~\(p_t^{CapB}=0\) but \(p_t^{ImmDetect}>0\)). These cases call for estimation of (or, in other words, accounting for the uncertainty in) the corresponding \(p_t^{ImmDetect}>0\). As long as such cases are relatively few, this can be accommodated by specifying a non-informative prior for the unknown value\footnote{For estimating only a subset of values within a vector/array (= partially observed variables), data (here \texttt{PropImmDetect{[}t{]}}) is provided including \texttt{NA} for unknown values and initial values are provided for the corresponding indeces (with initial values matching indeces of known data points being set to \texttt{NA}). Handling partially observed variables is discussed some more under \protect\hyperlink{ux5cux23ux5cux23ux5cux2520Imputationux5cux2520ofux5cux2520missingux5cux2520covariateux5cux2520values}{3.2.3 Imputation of missing covariate values}}:

\begin{Shaded}
\begin{Highlighting}[]
\DocumentationTok{\#\# Immigrant detection (= marking) probability}
\ControlFlowTok{for}\NormalTok{(t }\ControlFlowTok{in} \DecValTok{1}\SpecialCharTok{:}\NormalTok{Tmax)\{}
\NormalTok{  PropImmDetect[t] }\SpecialCharTok{\textasciitilde{}} \FunctionTok{dunif}\NormalTok{(}\DecValTok{0}\NormalTok{, }\DecValTok{1}\NormalTok{)}
\NormalTok{\}}
\end{Highlighting}
\end{Shaded}

The latent number of breeding immigrants, informed by data as outlined above, is estimated as pooled across age classes (\(ImmNoTot_t\)), but the since the population model in \texttt{SPI-IPM} is age structured \(ImmNoTot_t = ImmB_{Y,t} +ImmB_{A,t}\), where \(ImmB_{Y,t}\) and \(ImmB_{A,t}\) the numbers of breeding yearling and adult immigrants, respectively. These quantities are linked to the total numbers of immigrants in each age class (breeding and non-breeding), \(Imm_{a,t}\), that appear in the age-structured population model (\protect\hyperlink{ux5cux23ux5cux2520Openux5cux2520populationux5cux2520modelux5cux2520withux5cux25202ux5cux2520ageux5cux2520classes}{Chapter 2.1} through age-specific breeding probabilities.
By assuming equal breeding probabilities for locally recruited and newly immigrated females, we can use the breeding probabilities estimated via the mark-recapture likelihood (\(pB_{a,t}\)) to make the connection:
\begin{equation}
  ImmB_{a,t} \sim Binomial(Imm_{a,t}, pB_{a,t})
\end{equation}

In the code, the sub-model for immigration contains both the likelihood for the immigrant count data pooled across ages and the relationship between breeding immigrant females and all immigrant females in both age classes:

\begin{Shaded}
\begin{Highlighting}[]
\DocumentationTok{\#\# Likelihood for the number/age distribution of immigrant females}
\ControlFlowTok{for}\NormalTok{ (t }\ControlFlowTok{in} \DecValTok{2}\SpecialCharTok{:}\NormalTok{Tmax)\{}

  \CommentTok{\# Latent true number of breeding immigrants (count observation model)}
\NormalTok{  ImmNoObs[t] }\SpecialCharTok{\textasciitilde{}} \FunctionTok{dpois}\NormalTok{(}\FunctionTok{sum}\NormalTok{(ImmB[}\DecValTok{1}\SpecialCharTok{:}\NormalTok{A,t])}\SpecialCharTok{*}\NormalTok{PropImmDetect[t])}

  \CommentTok{\# Number of breeding immigrants per age class}
  \ControlFlowTok{for}\NormalTok{(a }\ControlFlowTok{in} \DecValTok{1}\SpecialCharTok{:}\NormalTok{A)\{}
\NormalTok{    ImmB[a,t] }\SpecialCharTok{\textasciitilde{}} \FunctionTok{dbin}\NormalTok{(pB[a,t], Imm[a,t])}
\NormalTok{  \}}
\NormalTok{\}}

\NormalTok{ImmNoObs[}\DecValTok{1}\NormalTok{] }\OtherTok{\textless{}{-}} \DecValTok{0}
\NormalTok{ImmB[}\DecValTok{1}\SpecialCharTok{:}\NormalTok{A,}\DecValTok{1}\NormalTok{] }\OtherTok{\textless{}{-}} \DecValTok{0}
\end{Highlighting}
\end{Shaded}

Note that various immigrant numbers are set to 0 at \(t=1\) since there is no way of distinguishing between locally-born and newly immigrated birds in the first year of study.

\hypertarget{priors-and-constraints}{%
\section{Priors and constraints}\label{priors-and-constraints}}

In the model description above, all vital rate parameters, as well as initial
population size and immigrant numbers, appear as fully time- and age-dependent
parameters (i.e.~indexed by \(t\) = year and \(a\) = age class). \texttt{SPI-IPM} is a
hierarchical model, meaning it also describes how demographic quantities vary
across time and age classes and requires priors for the parameters defining
these underlying processes. The priors in the model's standard implementation
are all non- or only weakly informative, but they can easily be replaced with
more informative priors if required (see e.g.~Chapter \protect\hyperlink{ux5cux23ux5cux2520Includingux5cux2520additionalux5cux2520dataux5cux2520andux5cux2520informativeux5cux2520priors}{8.2}).

Time- and age-dependence in all vital rates is expressed using generalized
mixed-effect models. The exact model structure for each vital rate,
including link function, is described in detail in Chapter \protect\hyperlink{ux5cux2520Modellingux5cux2520temporalux5cux2520variation}{3}. Irrespective of the vital rate, the basic model parameters fall into
three categories: (age-specific) intercepts representing time-average vital
rates (\(\mu\)), slope parameters for the effects of temporal covariates (\(\beta\)),
and standard deviations describing the distribution of random year effects
(\(\sigma\)).
The intercept parameters \(\mu\) are defined on the natural scale to allow for
more intuitive setting of priors. As such, non-informative Uniform(0,1) priors
can be used for the \(\mu\) of all vital rates representing probabilities
(\(\mu_a^{pB}\), \(\mu^{pNS}\), \(\mu_a^{sN}\), \(\mu^{sJ}\), and \(\mu^{sA}\)). For the
age-specific average clutch size (\(\mu_a^{CS}\)), we use a Uniform(0,10) prior.
This latter prior is not completely uninformative, as it sets an upper limit of
10 eggs for the average clutch size (this is reasonable for pied flycatchers
\textit{Ficedula hypoleuca} for which \texttt{SPI-IPM} was initially developed, but
may have to be adjusted for species that lay larger clutches).
All slope parameters \(\beta\) are given Uniform(-5,5) priors, and all \(\sigma\)
parameters (except for the one linked to immigration, \(\sigma_a^{Imm}\)) are
given Uniform(0,10) priors.

Priors and constraints are also required for the initial population size, i.e.
\(N_{Y,1}\) and \(N_{A,1}\) since the population process model only begins
predicting \(N\) from the second year (\(t=2\)) onward. Immigrant numbers have their
own set of priors (see below), and since \(N_{a,t} = localN_{a,t} + Imm_{a,t}\),
priors have to be set for \(localN_{a,1}\) specifically. \texttt{SPI-IPM} uses what is
essentially a discrete uniform prior for initial local population size:

\begin{Shaded}
\begin{Highlighting}[]
\DocumentationTok{\#\# Initial population sizes}

\ControlFlowTok{for}\NormalTok{(a }\ControlFlowTok{in} \DecValTok{1}\SpecialCharTok{:}\NormalTok{A)\{}
\NormalTok{  localN[a,}\DecValTok{1}\NormalTok{] }\SpecialCharTok{\textasciitilde{}} \FunctionTok{dcat}\NormalTok{(DU.prior[}\DecValTok{1}\SpecialCharTok{:}\NormalTok{N1.limit])}
\NormalTok{  N[a,}\DecValTok{1}\NormalTok{] }\OtherTok{\textless{}{-}}\NormalTok{ localN[a,}\DecValTok{1}\NormalTok{] }\SpecialCharTok{+}\NormalTok{ Imm[a,}\DecValTok{1}\NormalTok{]}
\NormalTok{\}}

\NormalTok{DU.prior[}\DecValTok{1}\SpecialCharTok{:}\NormalTok{N1.limit] }\OtherTok{\textless{}{-}} \DecValTok{1}\SpecialCharTok{/}\NormalTok{N1.limit}
\end{Highlighting}
\end{Shaded}

Here, \texttt{N1.limit} is a user-specified maximum possible number of local females in
any age class at the first time-step. The discrete uniform distribution is
approximated using a categorical distribution in which each integer number between
1 and \texttt{N1.limit} has the same probability (\texttt{1/N1.limit}) of occurring.
This approach is equivalent to using a regular Uniform(0,\texttt{N1.limit}) distribution
and subsequently rounding it.

The final set of priors and constraints in the model are for immigrant numbers
\(Imm_{Y,t}\) and \(Imm_{A,t}\). At the first time-step (\(t=1\)), immigrant numbers
are set to 0 (since there is no way of distinguishing previously local and
newly immigrated individuals in the first year of a study). Numbers at any
subsequent time-step are assumed to follow a 0-truncated normal distribution
with age-specific mean \(AvgImm_a\) (\texttt{AvgImm{[}a{]}}) and standard deviation
\(\sigma_a^{Imm}\) (\texttt{sigma.Imm{[}a{]})}):

\begin{Shaded}
\begin{Highlighting}[]
\DocumentationTok{\#\# Latent number of all immigrants (breeding \& non{-}breeding) per age class}
\NormalTok{Imm[}\DecValTok{1}\SpecialCharTok{:}\NormalTok{A,}\DecValTok{1}\NormalTok{] }\OtherTok{\textless{}{-}} \DecValTok{0}

\ControlFlowTok{for}\NormalTok{(t }\ControlFlowTok{in} \DecValTok{2}\SpecialCharTok{:}\NormalTok{Tmax)\{}
  \ControlFlowTok{for}\NormalTok{(a }\ControlFlowTok{in} \DecValTok{1}\SpecialCharTok{:}\NormalTok{A)\{}

\NormalTok{    ImmCont[a,t] }\SpecialCharTok{\textasciitilde{}} \FunctionTok{T}\NormalTok{(}\FunctionTok{dnorm}\NormalTok{(AvgImm[a], }\AttributeTok{sd =}\NormalTok{ sigma.Imm[a]), }\DecValTok{0}\NormalTok{, }\ConstantTok{Inf}\NormalTok{)}
\NormalTok{    Imm[a,t] }\OtherTok{\textless{}{-}} \FunctionTok{round}\NormalTok{(ImmCont[a,t])}
\NormalTok{  \}}
\NormalTok{\}}
\end{Highlighting}
\end{Shaded}

The rounding function ensures that immigrant numbers are integers.
Priors are then required for \(AvgImm_a\) and \(\sigma_a^{Imm}\), and these are
given as uniform distributions with values ranging from 0 to (a multiple of) a
user-specified maximum possible number of immigrant females in any age class:

\begin{Shaded}
\begin{Highlighting}[]
\ControlFlowTok{for}\NormalTok{(a }\ControlFlowTok{in} \DecValTok{1}\SpecialCharTok{:}\NormalTok{A)\{}
\NormalTok{  AvgImm[a] }\SpecialCharTok{\textasciitilde{}} \FunctionTok{dunif}\NormalTok{(}\DecValTok{0}\NormalTok{, AvgImm.limit)}
\NormalTok{  sigma.Imm[a] }\SpecialCharTok{\textasciitilde{}} \FunctionTok{dunif}\NormalTok{(}\DecValTok{0}\NormalTok{, AvgImm.limit}\SpecialCharTok{*}\DecValTok{10}\NormalTok{)}
\NormalTok{\}}
\end{Highlighting}
\end{Shaded}

\hypertarget{TempVar}{%
\chapter{Modelling temporal variation}\label{TempVar}}

Vital rates are expected to vary among years due to changes in environmental
conditions. \texttt{SPI-IPM} is set up to account for among-year variation in
different (age-specific) vital rates \(X_{a,t}\) using fixed effects of
supplied environmental covariates (\(cov1_t\), \(cov2_t\),\ldots) and year random effects (\(\epsilon_t^X\)). The resulting generalized linear
mixed-models for time- and age-specific vital rates therefore take the following
form:
\begin{equation}
  link(X_{a,t}) = link(\mu_a^X) + \beta_{cov1}^X\times cov1_t + \beta_{cov2}^X\times cov2_t + ... + \epsilon_t^X
\end{equation}
Here, \(\mu_a^X\) is the age-specific average vital rate (intercept) and \(\beta_{cov1}\)
and \(\beta_{cov2}\) are the slopes for the effects of covariates \(cov1\) and \(cov2\)
on the link scale, respectively.
The link function depends on the vital rate, and is set to logit for breeding
probabilities (\(pB_{a,t}\)), nest success probabilities (\(pNS_t\)), and survival
probabilities (\(sN_{a,t}\), \(sJ_t\), \(sA_t\)) and log for clutch size (\(CS_{a,t}\)).

\hypertarget{random-year-variation}{%
\section{Random year variation}\label{random-year-variation}}

The random year effects included in the basic implementation of \texttt{SPI-IPM} are
assumed to be normally distributed such that
\begin{equation}
  \epsilon_{t}^X \sim Normal(0, \sigma^X)
\end{equation}
where \(\sigma^X\) is the standard deviation of random year effects on vital rate
\(X_{a,t}\).
The random effects in the basic implementation are also age-independent, meaning
that \(\epsilon_t^X\) is included in the equations for both the yearling vital
rate \(X_{Y,t}\) and the adults vital rate \(X_{A,t}\). The exception are the annual
survival rates for juveniles (\(sJ_t\)) and (\(sA_t\)), which both have separate
random effects because the drivers of survival variation are expected to vary
between those two age classes, and the basic implementation includes an
additional covariate effect \(sJ_t\) only\footnote{Note that the implementation of this
  in \texttt{IPMSetup.R} looks slightly different as it uses a vectorized formulation
  (calculations done over all time-steps simultaneously instead of using a
  for-loop). Vectorized calculations are a nifty feature available in NIMBLE (but
  not BUGS and JAGS); more on this in Chapter \protect\hyperlink{ux5cux23ux5cux2520Efficientux5cux2520implementationux5cux2520usingux5cux2520NIMBLE}{4.1}.}:

\begin{Shaded}
\begin{Highlighting}[]
\ControlFlowTok{for}\NormalTok{(t }\ControlFlowTok{in} \DecValTok{1}\SpecialCharTok{:}\NormalTok{Tmax)\{}
    
  \DocumentationTok{\#\# Age{-} and time{-}dependent survival probabilities}
  \FunctionTok{logit}\NormalTok{(sJ[t]) }\OtherTok{\textless{}{-}} \FunctionTok{logit}\NormalTok{(Mu.sJ) }\SpecialCharTok{+}\NormalTok{ beta3.sJ}\SpecialCharTok{*}\NormalTok{cov3[t] }\SpecialCharTok{+}\NormalTok{ epsilon.sJ[t]}
  \FunctionTok{logit}\NormalTok{(sA[t]) }\OtherTok{\textless{}{-}} \FunctionTok{logit}\NormalTok{(Mu.sA) }\SpecialCharTok{+}\NormalTok{ epsilon.sA[t]}

  \DocumentationTok{\#\# Temporal random effects}
\NormalTok{    epsilon.sJ[t] }\SpecialCharTok{\textasciitilde{}} \FunctionTok{dnorm}\NormalTok{(}\DecValTok{0}\NormalTok{, }\AttributeTok{sd =}\NormalTok{ sigma.sJ)}
\NormalTok{    epsilon.sA[t] }\SpecialCharTok{\textasciitilde{}} \FunctionTok{dnorm}\NormalTok{(}\DecValTok{0}\NormalTok{, }\AttributeTok{sd =}\NormalTok{ sigma.sA)}
\NormalTok{\}}
\end{Highlighting}
\end{Shaded}

All random effects are treated as independent (= not correlated) in the basic
implementation of the model, but the inclusion of correlation of random effects
across age-classes and/or vital rates is straightforward to implement\footnote{Wheter or
  not formally including random effects correlations is useful or not depends on
  the biological questions of interest and the amoung of data available. When
  testing a model with correlated random effects for juvenile and adult survival
  on seven datasets from breeding populations of pied flycatchers in the UK, I
  found that estimates did not differ from those obtained from a model with
  independent random effects, and the posterior distribution for the correlation
  coefficient was so wide that no inference on strength or direction of the
  correlation was possible.} using e.g.
multivariate normal distributions or approaches similar to the one used in
\citet{nater2020}.

\hypertarget{temporal-covariates}{%
\section{Temporal covariates}\label{temporal-covariates}}

The basic implementation of \texttt{SPI-IPM} features an example covariate model
structure that was motivated by an analysis of populations of pied flycatcher
(\textit(Ficedula hypoleuca)) breeding in the UK. It involves three different
covariates (\(cov1\), \(cov2\), and \(cov3\)) that are assumed to affect nest success
probability (\(pNS_{t}\)), nestling survival (\(sN_{a,t}\)), and juvenile survival
(\(sJ_t\)). However, it is very straightforward to alter the code to fit whatever
alternative covariate structure is suitable for your particular analysis since
the inclusion different/additional continuous and categorical covariates always
works according to the same principles \citep[see also][]{kery2011}. All covariates are
need to be passed to \texttt{SPI-IPM} as vectors or arrays.

\hypertarget{continuous-variables}{%
\subsection{Continuous variables}\label{continuous-variables}}

Continuous variables are included as covariate effects using a specific slope
parameter (\(\beta\)). Temporal effects then take the form
\(\beta_{cov1}\times cov1_t\), where \(\beta_{cov1}\) is an estimated parameter
quantifying strength and direction of the effect and \(cov1_t\) is the value of
covariate \(cov1\) at time \(t\).
In (generalized) linear mixed effects models as they are used in \texttt{SPI-IPM},
effects of several different covariates can just be added up on the link scale.
For \(pNS_t\), for example, this codes as\footnote{see Footnote 1}:

\begin{Shaded}
\begin{Highlighting}[]
\ControlFlowTok{for}\NormalTok{(t }\ControlFlowTok{in} \DecValTok{1}\SpecialCharTok{:}\NormalTok{Tmax)\{}
  \FunctionTok{logit}\NormalTok{(pNS[t]) }\OtherTok{\textless{}{-}} \FunctionTok{logit}\NormalTok{(Mu.pNS) }\SpecialCharTok{+}\NormalTok{ beta1.pNS}\SpecialCharTok{*}\NormalTok{cov1[t] }\SpecialCharTok{+}\NormalTok{ beta2.pNS}\SpecialCharTok{*}\NormalTok{cov2[t] }\SpecialCharTok{+}\NormalTok{ epsilon.pNS[t]}
\NormalTok{\}}
\end{Highlighting}
\end{Shaded}

The basic implementation of \texttt{SPI-IPM} includes the following covariate models:
\begin{align*}
  & logit(pNS_t) = logit(\mu^{pNS}) + \beta_{cov1}^{pNS}\times cov1_t + \beta_{cov2}^{pNS} \times cov2_t + \epsilon_t^{pNS} \\
  & logit(sN_{a,t}) = logit(\mu_a^{pNS}) + \beta_{cov1}^{sN}\times cov1_t + \beta_{cov2}^{sN} \times cov2_t + \epsilon_t^{sN} \\
  & logit(sJ_t) = logit(\mu^{sJ}) + \beta_{cov3}^{sJ}\times cov3_t + \epsilon_t^{sJ}
\end{align*}
All covariates are continuous annual variables that have been standardized and
centered (mean = 0, sd = 1) prior to analysis.
\(cov1\) and \(cov2\) represent environmental conditions during the incubation and
nestling period and hence influence nest success (\(pNS_t\)) and nestling survival
(\(sN_{a,t}\)). In the case of the latter, covariates are further assumed to have
the same magnitude of effect on the nests of yearling and adult females (i.e.
the \(\beta\) parameters are independent of age). \(cov3\), on the other hand,
symbolizes environmental conditions after fledging which impact juvenile annual
survival (\(sJ_t\)). No covariate effects are included for the other vital rates.

\hypertarget{categorical-variables}{%
\subsection{Categorical variables}\label{categorical-variables}}

The basic \texttt{SPI-IPM} does not include any categorical covariates, but since such
covariates may be relevant to a wide range of questions (e.g.~some of the points
raised in Chapter \protect\hyperlink{ux5cux23ux5cux2520Adaptingux5cux2520theux5cux2520populationux5cux2520modelux5cux2520forux5cux2520yourux5cux2520speciesux2fpopulation}{8.2}),
I briefly illustrate how they could be included into vital rate models.

Generally, there are two approaches to modelling categorical covariates in this
context.

The first approach works analogous to the approach for continuous covariates,
i.e.~it uses the form \(\beta_{cov}\times cov_t\). This is most relevant for
binary categorical covariates that symbolize some sort of ``on-off'' process. An
example of this would be if you would like to model the effect of an experimental
treatment that has been performed in some years (\(cov_t = 1\)) but not others
(\(cov_t = 0\)). Your binary covariate then works as a ``switch'' that determines
whether or not the effect of the experimental treatment (\(\beta_{cov}\)) applied
in a given year \(t\) or not since \(link(X_t) = link(\mu^X) + \beta_{cov_t}\times cov_t\) becomes \(link(\mu^X) + \beta_{cov_t}\) when \(cov_t = 1\) and \(link(\mu^X)\)
when \(cov_t = 0\).

The second approach works via (nested) indexing and is more flexible since it
can technically account for any number of levels in your categorical covariate.
This can be relevant, for example, for categories of years (``good'', ``average'',
``bad''), habitat types (``deciduous forest'', ``coniferous forest''), or individuals
(``male'', ``female'').
The approach still uses \(\beta\) parameters, but instead of multiplying the
\(\beta\) with the covariate value, we index the \(\beta\) by the covariate value
such that \(\beta_1\) corresponds to the effect of category 1 (\(cat = 1\)),
\(\beta_2\) corresponds to the effect of category 2 (\(cat = 2\)), and so on:

\begin{equation}
  link(X_{cat}) = link(\mu^X) + \beta_{cat}
\end{equation}
Priors then need to be provided for each category-specific \(\beta\).

In practice, \texttt{SPI-IPM} still requires vital rates \(X\) to be indexed by age class
and year (at least with the population model described in Chapter \protect\hyperlink{ux5cux2520IPMux5cux2520Construction}{2}).
That's where nested indexing becomes relevant.
The relationship of a vital rate \(X_{a,t}\) with a categorical year covariate can,
for example, be coded as follows:

\begin{Shaded}
\begin{Highlighting}[]
\ControlFlowTok{for}\NormalTok{(a }\ControlFlowTok{in} \DecValTok{1}\SpecialCharTok{:}\NormalTok{A)\{}
  \ControlFlowTok{for}\NormalTok{(t }\ControlFlowTok{in} \DecValTok{1}\SpecialCharTok{:}\NormalTok{Tmax)\{}
      \FunctionTok{log}\NormalTok{(X[a,t]) }\OtherTok{\textless{}{-}} \FunctionTok{log}\NormalTok{(Mu.X[a]) }\SpecialCharTok{+}\NormalTok{ beta[cov[t]]}
\NormalTok{  \}}
\NormalTok{  Mu.X }\SpecialCharTok{\textasciitilde{}} \FunctionTok{dunif}\NormalTok{(}\DecValTok{0}\NormalTok{, }\DecValTok{10}\NormalTok{)}
\NormalTok{  beta }\SpecialCharTok{\textasciitilde{}} \FunctionTok{dunif}\NormalTok{(}\SpecialCharTok{{-}}\DecValTok{5}\NormalTok{, }\DecValTok{5}\NormalTok{)}
\NormalTok{\}}
\end{Highlighting}
\end{Shaded}

where \texttt{cov{[}t{]}} is a vector of integer numbers that represent the different year
categories.

Introducing categorical effects that rely on additional structure beyond year
and age (for example effects of sex or location) requires changing the underlying
population model. Such extensions are currently not implemented in \texttt{SPI-IPM},
but see Chapter \protect\hyperlink{ux5cux2520Usefulux5cux2520extensionsux5cux2520andux5cux2520outlook}{8} for some perspectives.

\hypertarget{imputation-of-missing-covariate-values}{%
\subsection{Imputation of missing covariate values}\label{imputation-of-missing-covariate-values}}

Perhaps you have been wondering about how to deal with NAs in your covariate
data? The good news is that \texttt{SPI-IPM} (just like any other Bayesian
hierarchical model) can accommodate NAs in both continuous and categorical
covariates. The (perhaps) less good news is that how well it works really
depends on how large a proportion of your covariate data is NA.

There are three practical requirements for working with partially observed
covariate data:

\begin{enumerate}
\def\labelenumi{\arabic{enumi}.}
\item
  Your covariate data containing numbers for your observed covariate values and NAs for your unobserved/unknown covariate values
\item
  Initial values with the same dimensions as your covariate data containing
  numbers in the positions of NA covariate values and NAs in the positions of
  observed covariate values.
\item
  A model describing the distribution of missing covariate values.
\end{enumerate}

Numbers 1. and 2. are pretty self-explanatory (but see Chapter \protect\hyperlink{ux5cux23ux5cux2520Simulationux5cux2520ofux5cux2520initialux5cux2520values}{4.2} for more information on sampling initial values).
Number 3. is going to depend on what type of covariate data you are dealing with.

The basic implementation of \texttt{SPI-IPM} is set up to be able to deal with NA
values in the continuous temporal covariates \(cov1\), \(cov2\), and \(cov3\). These
covariates are assumed to have been standardized and centered, i.e.~they should
more or less follow a \(Normal(mean= 0, sd = 1)\) distribution. If we assume that
the observed and unobserved covariate values follow the same distribution (i.e.
the missing values are a random subset of all values), this can be used to
specify the process model for the missing covariates in the code:

\begin{Shaded}
\begin{Highlighting}[]
\ControlFlowTok{for}\NormalTok{(t }\ControlFlowTok{in} \DecValTok{1}\SpecialCharTok{:}\NormalTok{Tmax)\{}
\NormalTok{  cov1[t] }\SpecialCharTok{\textasciitilde{}} \FunctionTok{dnorm}\NormalTok{(}\DecValTok{0}\NormalTok{, }\AttributeTok{sd =} \DecValTok{1}\NormalTok{)}
\NormalTok{  cov2[t] }\SpecialCharTok{\textasciitilde{}} \FunctionTok{dnorm}\NormalTok{(}\DecValTok{0}\NormalTok{, }\AttributeTok{sd =} \DecValTok{1}\NormalTok{)}
\NormalTok{  cov3[t] }\SpecialCharTok{\textasciitilde{}} \FunctionTok{dnorm}\NormalTok{(}\DecValTok{0}\NormalTok{, }\AttributeTok{sd =} \DecValTok{1}\NormalTok{)}
\NormalTok{\}}
\end{Highlighting}
\end{Shaded}

There are numerous alternatives for specifying distributions of missing values
in continous covariates, and they can be accommodated by changing the above
section in the code.

The number of candidate distributions are a bit more limited when there are
missing values in categorical covariates.
Chapter \protect\hyperlink{ux5cux23ux5cux23ux5cux2520Includingux5cux2520partiallyux5cux2520observedux5cux2520ageux5cux2520information}{8.2.1} outlines an
example for dealing with partially missing information on individual age. It
may also be helpful to remember that models with partially observed categorical
variables are essentially ``mixture models'' including auxiliary data about the
underlying distribution.

\hypertarget{notes-on-covariate-selection}{%
\section{Notes on covariate selection}\label{notes-on-covariate-selection}}

To be added later.

\hypertarget{IPMImp}{%
\chapter{IPM Implementation}\label{IPMImp}}

\hypertarget{efficient-implementation-using-nimble}{%
\section{Efficient implementation using NIMBLE}\label{efficient-implementation-using-nimble}}

\texttt{SPI-IPM} is implemented in a Bayesian framework and, more specifically, it is
fit using NIMBLE \citep{devalpine2017}. In many ways, NIMBLE is the successor to
BUGS and JAGS, using very similar syntax but offering higher efficiency, much
more flexibility, and a range of smaller ``quality-of-life improvements''.
Unlike its predecessors, NIMBLE is not a software on its own and is installed
via the R-package \texttt{nimble} \citep{nimbleR}. A lot of resources on setting up and
working with NIMBLE are available on the \href{https://r-nimble.org/}{NIMBLE website}
and this manual will not give a comprehensive overview of NIMBLE or the
\texttt{nimble} R package.
In the following, I will instead briefly outline some of
NIMBLE's features that \texttt{SPI-IPM} makes use of, and that may be good to be aware
of particularly for people switching over from BUGS/JAGS. I will also briefly
outline the structure for the basic call to NIMBLE to run a model.

\hypertarget{alternative-specification-of-distributions}{%
\subsection{Alternative specification of distributions}\label{alternative-specification-of-distributions}}

First, NIMBLE allows alternative specifications for statistical distributions
(see the \href{https://r-nimble.org/manuals/NimbleUserManual.pdf}{NIMBLE manual}
for an overview). This is particularly convenient for some commonly used
distributions such as the normal distribution, which had to be specified via
a mean and a precision (\(\tau\)) in BUGS/JAGS. The latter was then often related
to the (some might say) more useful standard deviation (\(\sigma\)):

\begin{Shaded}
\begin{Highlighting}[]
\ControlFlowTok{for}\NormalTok{(t }\ControlFlowTok{in} \DecValTok{1}\SpecialCharTok{:}\NormalTok{Tmax)\{}
\NormalTok{  epsilon[t] }\SpecialCharTok{\textasciitilde{}} \FunctionTok{dnorm}\NormalTok{(}\DecValTok{0}\NormalTok{, tau)}
\NormalTok{\}}
\NormalTok{tau }\OtherTok{\textless{}{-}} \FunctionTok{pow}\NormalTok{(sigma, }\SpecialCharTok{{-}}\DecValTok{2}\NormalTok{)}
\NormalTok{sigma }\SpecialCharTok{\textasciitilde{}} \FunctionTok{dunif}\NormalTok{(}\DecValTok{0}\NormalTok{, }\DecValTok{5}\NormalTok{)}
\end{Highlighting}
\end{Shaded}

Looks familiar?\\
NIMBLE allows to specify the distribution parameters exactly as above, but also
offers alternatives such as parameterisation by mean and standard deviation
directly, e.g.:

\begin{Shaded}
\begin{Highlighting}[]
\ControlFlowTok{for}\NormalTok{(t }\ControlFlowTok{in} \DecValTok{1}\SpecialCharTok{:}\NormalTok{Tmax)\{}
\NormalTok{  epsilon[t] }\SpecialCharTok{\textasciitilde{}} \FunctionTok{dnorm}\NormalTok{(}\DecValTok{0}\NormalTok{, }\AttributeTok{sd =}\NormalTok{ sigma)}
\NormalTok{\}}
\NormalTok{sigma }\SpecialCharTok{\textasciitilde{}} \FunctionTok{dunif}\NormalTok{(}\DecValTok{0}\NormalTok{, }\DecValTok{5}\NormalTok{)}
\end{Highlighting}
\end{Shaded}

\hypertarget{vectorized-calculations}{%
\subsection{Vectorized calculations}\label{vectorized-calculations}}

A second useful feature is vectorized calculation of deterministic nodes. At
some point, everyone has heard that when programming in \texttt{R}, vectorization
outperforms for-loops when it comes to efficiency and speed. In many cases, this
is also true for MCMC, and NIMBLE therefore supports vectorization of
calculations for deterministic nodes (i.e.~nodes assigned via \texttt{\textless{}-} in the code).
\texttt{SPI-IPM} uses vectorized calculations instead of for-loops in a variety of
cases, for example for defining the models underlying temporal variation in
vital rates:

\begin{Shaded}
\begin{Highlighting}[]
\FunctionTok{logit}\NormalTok{(pNS[}\DecValTok{1}\SpecialCharTok{:}\NormalTok{Tmax]) }\OtherTok{\textless{}{-}} \FunctionTok{logit}\NormalTok{(Mu.pNS) }\SpecialCharTok{+}\NormalTok{ beta1.pNS}\SpecialCharTok{*}\NormalTok{cov1[}\DecValTok{1}\SpecialCharTok{:}\NormalTok{Tmax] }\SpecialCharTok{+}\NormalTok{ beta2.pNS}\SpecialCharTok{*}\NormalTok{cov2[}\DecValTok{1}\SpecialCharTok{:}\NormalTok{Tmax] }\SpecialCharTok{+}\NormalTok{ epsilon.pNS[}\DecValTok{1}\SpecialCharTok{:}\NormalTok{Tmax]}
\end{Highlighting}
\end{Shaded}

instead of

\begin{Shaded}
\begin{Highlighting}[]
\ControlFlowTok{for}\NormalTok{(t }\ControlFlowTok{in} \DecValTok{1}\SpecialCharTok{:}\NormalTok{Tmax)\{}
  \FunctionTok{logit}\NormalTok{(pNS[t]) }\OtherTok{\textless{}{-}} \FunctionTok{logit}\NormalTok{(Mu.pNS) }\SpecialCharTok{+}\NormalTok{ beta1.pNS}\SpecialCharTok{*}\NormalTok{cov1[t] }\SpecialCharTok{+}\NormalTok{ beta2.pNS}\SpecialCharTok{*}\NormalTok{cov2[t] }\SpecialCharTok{+}\NormalTok{ epsilon.pNS[t]}
\NormalTok{\}}
\end{Highlighting}
\end{Shaded}

It is important to note that vectorization works for deterministic nodes, but
-- per today -- not for stochastic nodes (i.e.~nodes assigned via \texttt{\textasciitilde{}}).

\hypertarget{custom-distributions}{%
\subsection{Custom distributions}\label{custom-distributions}}

The third class of NIMBLE features that \texttt{SPI-IPM} capitalizes on is the ability
to define custom distributions. Specifically, \texttt{SPI-IPM} uses a custom distribution
in the likelihood for the mark-recapture data.\\
In most IPMs, the analysis of mark-recapture data to estimate survival represents
the bottleneck for MCMC efficiency: sampling hundreds -- if not thousands -- of
latent alive or dead states is computationally expensive and results in long
MCMC runtimes \citep{gimenez2007}.
Summarising individual capture histories into ``m-arrays'' has long been the only
way to reduce runtimes of Bayesian mark-recapture models \citep{kery2011}, but many
find this format less intuitive and it quickly becomes convoluted and impractical
when working with predictors/groups other than age class. With the rise of
NIMBLE, \citet{turek2016} developed an approach to defining marginalized likelihoods
that integrate over latent states and lead to tremendous increases in MCMC
efficiency. Basic implementations of the marginalized likelihood for mark-recapture
models were included in the \texttt{nimbleEcology} R package \citep{nimbleEcol}.\\
\texttt{SPI-IPM} uses an extended version of the \texttt{dCJS\_vv} distribution contained in
\texttt{nimbleEcology} which not only integrates over latent states, but also runs on
only unique -- instead of all -- capture histories \citep[analogous to the goose
example in][]{turek2016}. Preliminary tests have shown that mark-recapture models
with a likelihood specified using this distribution (\texttt{dCJS\_vv\_sum}) have very
similar runtimes as implementations using m-arrays for small to medium-sized
datasets, and outspeed m-array formulations in for larger datasets.\\
The specification of the custom distribution is contained in \texttt{dCJS\_CustomDist.R}
and its implementation within the model code looks quite minimalistic:

\begin{Shaded}
\begin{Highlighting}[]
\DocumentationTok{\#\# Likelihood with custom distribution}
\ControlFlowTok{for}\NormalTok{ (i }\ControlFlowTok{in} \DecValTok{1}\SpecialCharTok{:}\NormalTok{n.CH)\{}
\NormalTok{  y.sum[i, first.sum[i]}\SpecialCharTok{:}\NormalTok{last.sum[i]] }\SpecialCharTok{\textasciitilde{}} \FunctionTok{dCJS\_vv\_sum}\NormalTok{(}
       \AttributeTok{probSurvive =}\NormalTok{ phi.CH[i, first.sum[i]}\SpecialCharTok{:}\NormalTok{last.sum[i]],}
       \AttributeTok{probCapture =}\NormalTok{ p.CH[i, first.sum[i]}\SpecialCharTok{:}\NormalTok{last.sum[i]],}
       \AttributeTok{len =}\NormalTok{ last.sum[i]}\SpecialCharTok{{-}}\NormalTok{first.sum[i]}\SpecialCharTok{+}\DecValTok{1}\NormalTok{,}
       \AttributeTok{mult =}\NormalTok{ CHs.count[i])}
\NormalTok{\}}
\end{Highlighting}
\end{Shaded}

\texttt{y.sum} is a matrix containing all \texttt{n.CH} unique capture histories and the
likelihood is fit for the range of observations between the first capture of an
individual (\texttt{first.sum}) and the last possible time this individual could still
have been alive\footnote{in practice this can be set to e.g.~year of the last recorded
  observation of the individual plus maximum lifesspan, plus some. In my analysis
  of UK-breeding pied flycatchers, for example, I have set it -- generously -- to
  20 years after the last recorded capture.}(\texttt{last.sum}). \texttt{probSurvive} and
\texttt{probCapture} are vectors containing the survival and recapture probabilities
relevant for the capture history (see \texttt{IPMSetup.R} and Chapter \protect\hyperlink{ux5cux23ux5cux23ux5cux2520Mark-recaptureux5cux2520dataux5cux2520likelihood}{2.2.4} for details). The argument \texttt{len} quantifies the length of \texttt{y.sum}
and is required for successful compilation of the custom function. Finally, \texttt{mult}
is an integer number specifying how many individual birds shared capture history
\texttt{i}.

\hypertarget{calling-nimble-from-r}{%
\subsection{Calling NIMBLE from R}\label{calling-nimble-from-r}}

The simplest way to implement and run a Bayesian model in NIMBLE is via the
wrapper-function \texttt{nimbleMCMC}. The structure of the \texttt{nimbleMCMC} call is very
similar to the functions provided with the different R-packages for running
BUGS/JAGS and -- for \texttt{SPI-IPM} -- looks like this (code line 49 in \texttt{IPMRun\_PopID.R}):

\begin{Shaded}
\begin{Highlighting}[]
\NormalTok{SPI.IPM }\OtherTok{\textless{}{-}} \FunctionTok{nimbleMCMC}\NormalTok{(}\AttributeTok{code =}\NormalTok{ SPI.IPMcode, }\AttributeTok{constants =}\NormalTok{ SPI.IPMconstants, }\AttributeTok{data =}\NormalTok{ SPI.IPMdata, }\AttributeTok{inits =}\NormalTok{ Inits, }\AttributeTok{monitors =}\NormalTok{ parameters, }\AttributeTok{niter =}\NormalTok{ ni, }\AttributeTok{nburnin =}\NormalTok{ nb, }\AttributeTok{nchains =}\NormalTok{ nc, }\AttributeTok{thin =}\NormalTok{ nt, }\AttributeTok{setSeed =}\NormalTok{ mySeed, }\AttributeTok{samplesAsCodaMCMC =} \ConstantTok{TRUE}\NormalTok{)}
\end{Highlighting}
\end{Shaded}

\texttt{code} is the model code, formatted as \texttt{nimbleCode}. \texttt{constants} and \texttt{data}
contain all information relevant for parameterising the model, as well as all
the observational data (see Chapter \protect\hyperlink{ux5cux23ux5cux2520Organisationux5cux2520forux5cux2520analysisux5cux2520withux5cux2520NIMBLE}{1.4}).
\texttt{inits} is a list of initial values that is described further below, and \texttt{monitors}
is a character vector containing the names of all paramters that should be
monitored. \texttt{niter}, \texttt{nburnin}, \texttt{nchains}, and \texttt{thin} are the numbers of iterations,
samples to discard as burn-in, number of chains, and thinning interval for the
MCMC. \texttt{setSeed} is absolutely crucial! By setting the seed within the call to
NIMBLE, your entire MCMC becomes reproducible which is invaluable also for
trouble-shooting. The final argument, \texttt{samplesAsCodaMCMC}, is for people like
me who like to get back raw samples and make their own summaries instead of
having the function return larger objects containing the samples along with
a variety of summaries and other stats.

\hypertarget{simulation-of-initial-values}{%
\section{Simulation of initial values}\label{simulation-of-initial-values}}

As a general rule, all parameters that have priors provided for them within
Bayesian models also require initial values. For \texttt{SPI-IPM}, this applies to:

\begin{itemize}
\tightlist
\item
  Vital rate averages (\(\mu\) /\texttt{Mu})
\item
  Environmental effects on vital rates (\(\beta\) /\texttt{beta})
\item
  Standard deviation for year random effects on vital rates (\(\sigma\) /\texttt{sigma})
\item
  Initial population sizes (\texttt{localN{[},1{]}})
\item
  Average and standard deviation of immigrant numbers (\texttt{AvgImm}, \texttt{sigma.Imm})
\item
  Missing covariate values (subset of \texttt{cov1}, \texttt{cov2}, \texttt{cov3})
\item
  Missing observation probabilities (subset of \texttt{PropImmDetect})
\end{itemize}

Provided that initial values for all of these parameters have been provided,
NIMBLE is able to calculate initial values for all of the remaining downstream
nodes using the realtionships specified in the model code. In practice, however,
the more different data sources and sub-models are included in an integrated
model, the more likely it is that this ``automatic'' initialization of nodes will
result in conflicts, i.e.~initial values that are not compatible with each other.

While it may work for a single IPM to re-sample initial values repeatedly until
a set without conflicts has been found by chance, this is vastly impractical for
a framework like \texttt{SPI-IPM} which is designed to also work well for comparative
analyses, i.e.~may be run on several datasets/populations at a time.\\
For that reason, manual initialization of all nodes makes sense for \texttt{SPI-IPM} and
is implemented in the function \texttt{SPI\_IPM.inits} (in file \texttt{InitSim.R}).

The function runs a simulation of the entire population model, including the
reconstruction of missing covariate and detection parameter values in four
steps:

\begin{enumerate}
\def\labelenumi{\arabic{enumi}.}
\tightlist
\item
  Simulation of missing covariate values
\item
  Simulation of age- and time-specific vital rates
\item
  Simulation of initial population size and population trajectory over the study period, incl.~immigration
\item
  Simulation of missing detection parameters
\end{enumerate}

The elements passed to the function are the collated data and constants
(\texttt{IPM.data} and \texttt{IPM.constants}, see Chapter \protect\hyperlink{ux5cux23ux5cux2520Organisationux5cux2520forux5cux2520analysisux5cux2520withux5cux2520NIMBLE}{1.4}),
as well as a logical argument determining whether initial values for year random
effects should be sampled from their simulated distribution or set to 0 (\texttt{sampleRE}).

To sample initial values for several chains, the function is called multiple
times within a list.

\begin{Shaded}
\begin{Highlighting}[]
\NormalTok{Inits }\OtherTok{\textless{}{-}} \FunctionTok{list}\NormalTok{(}
  \FunctionTok{SPI\_IPM.inits}\NormalTok{(}\AttributeTok{IPM.data =}\NormalTok{ SPI.IPMdata, }\AttributeTok{IPM.constants =}\NormalTok{ SPI.IPMconstants, }\AttributeTok{sampleRE =} \ConstantTok{FALSE}\NormalTok{),}
  \FunctionTok{SPI\_IPM.inits}\NormalTok{(}\AttributeTok{IPM.data =}\NormalTok{ SPI.IPMdata, }\AttributeTok{IPM.constants =}\NormalTok{ SPI.IPMconstants, }\AttributeTok{sampleRE =} \ConstantTok{FALSE}\NormalTok{))}
\end{Highlighting}
\end{Shaded}

is an example for initializing two chains.

Note that I am sampling the initial values and storing them in an object called
\texttt{Inits} in the example, which I then pass to \texttt{nimbleMCMC} afterwards, instead of
calling the \texttt{SPI\_IPM.inits} directly within the call to \texttt{nimbleMCMC}. The reason
for ``pre-sampling'' initial values in this way is that it makes trouble-shooting
much easier: if initialization problems occur at specific nodes, I can
investigate both the associated data AND pre-sampled initial values to work out
the cause of the discrepancy.\\
For the purpose of facilitating trouble-shooting, but also to promote reproducibility,
it is also good practice to set a seed prior to pre-sampling initial values
(in \texttt{IPMRun\_PopID.R}, for example, the seed is set right at the start of the
script).

\hypertarget{test-runs-and-full-runs-chains-iterations-burn-in-and-thinning}{%
\section{Test runs and full runs: chains, iterations, burn-in, and thinning}\label{test-runs-and-full-runs-chains-iterations-burn-in-and-thinning}}

If you have ample experience working with Bayesian models and MCMC, you can jump
over this section.

The following is a (perhaps obvious) practical tip for users that are newer to
Bayesian modelling and/or IPMs: always do a test-run of your implementation to
make sure it works before running a long MCMC!

In the original code, \texttt{SPI-IPM} is set up for a full run of 4 chains with
200 000 iterations each (see code lines 26-29 in \texttt{IPMRun\_PopID.R}). The first
50 000 iterations of each chain are discarded as burn-in and the remainder
thinned by 30, resulting in a combined posterior consisting of 4\(\times\) 5000 = 20 000
samples. This is rather generous, and it is not unlikely that your analysis will
require substantially less iterations to reach convergence (see Chapter
\protect\hyperlink{ux5cux23ux5cux2520Assessingux5cux2520chainux5cux2520convergence}{5.1} for how to assess convergence).
Nonetheless, running the full MCMC can easily take several hours and ideally,
you would discover issues such as bad initialization or mistakes made during
code adjustments \textbf{before} having to wait that long.\\
That's where short test runs become useful. The shortest MCMC you can run is
two iterations long, no burn-in, and with thinning interval 1 (= no thinning).
By running that first, you can check that your model builds smoothly and contains
no unwanted NAs, and that initialization works as it should creates no conflicts.
With slightly more iterations in a test chain, you can also make sure that all
nodes that should be updating are in fact updating. Most importantly, this will
allow you to test your model's setup and trouble-shoot some basic implementation
issues without having to wait for hours.

\hypertarget{trouble-shooting-implementation-issues}{%
\section{Trouble-shooting implementation issues}\label{trouble-shooting-implementation-issues}}

Specific content will be added later. In the meantime, a lot of helpful
information can be found in the materials on the \href{https://r-nimble.org/}{NIMBLE website}
and via the Google group \href{https://groups.google.com/g/nimble-users}{``nimble-users''}.

For issues that are related specifically to \texttt{SPI-IPM}, you can also get in touch
via \href{mailto:chloe.nater@nina.no}{email}.

\hypertarget{ModelAssm}{%
\chapter{Model Assessment}\label{ModelAssm}}

\hypertarget{assessing-chain-convergence}{%
\section{Assessing chain convergence}\label{assessing-chain-convergence}}

\hypertarget{plotting-data-vs.-predictions}{%
\section{Plotting data vs.~predictions}\label{plotting-data-vs.-predictions}}

\hypertarget{comparing-estimates-from-integrated-vs.-independent-analyses}{%
\section{Comparing estimates from integrated vs.~independent analyses}\label{comparing-estimates-from-integrated-vs.-independent-analyses}}

\hypertarget{reality-check-using-stochastic-simulations}{%
\section{``Reality check'' using stochastic simulations}\label{reality-check-using-stochastic-simulations}}

\hypertarget{other-approaches}{%
\section{Other approaches}\label{other-approaches}}

Running for additional years and comparing to non-included data, PPCs, etc.

\hypertarget{ResultsViz}{%
\chapter{Visualizing and interpreting direct IPM outputs}\label{ResultsViz}}

TBA

\hypertarget{population-trajectories}{%
\section{Population trajectories}\label{population-trajectories}}

TBA

\hypertarget{within-population-variation-in-vital-rates}{%
\section{Within-population variation in vital rates}\label{within-population-variation-in-vital-rates}}

TBA

\hypertarget{age-class-specific-averages}{%
\subsection{Age-class-specific averages}\label{age-class-specific-averages}}

TBA

\hypertarget{year-by-year-variation}{%
\subsection{Year-by-year variation}\label{year-by-year-variation}}

TBA

\hypertarget{between-population-variation-in-vital-rates}{%
\section{Between-population variation in vital rates}\label{between-population-variation-in-vital-rates}}

TBA

\hypertarget{population-specific-averages}{%
\subsection{Population-specific averages}\label{population-specific-averages}}

TBA

\hypertarget{year-by-year-variation-1}{%
\subsection{Year-by-year variation}\label{year-by-year-variation-1}}

TBA

\hypertarget{covariate-effects}{%
\section{Covariate effects}\label{covariate-effects}}

TBA

\hypertarget{AddAnalyses}{%
\chapter{Follow-up Analyses}\label{AddAnalyses}}

\hypertarget{testing-for-time-trends}{%
\section{Testing for time-trends}\label{testing-for-time-trends}}

TBA

\hypertarget{testing-for-density-dependence}{%
\section{Testing for density-dependence}\label{testing-for-density-dependence}}

TBA

\hypertarget{investigating-cross-population-covariation}{%
\section{Investigating cross-population covariation}\label{investigating-cross-population-covariation}}

TBA

\hypertarget{quantifying-demographic-contributions-to-short-term-population-dynamics}{%
\section{Quantifying demographic contributions to short term population dynamics}\label{quantifying-demographic-contributions-to-short-term-population-dynamics}}

TBA

\hypertarget{year-by-year-variation-in-population-growth-rate-random-design-ltre}{%
\subsection{Year-by-year variation in population growth rate (random design LTRE)}\label{year-by-year-variation-in-population-growth-rate-random-design-ltre}}

TBA

\hypertarget{year-to-year-differences-in-population-growth-rate-fixed-design-ltre}{%
\subsection{Year-to-year differences in population growth rate (fixed design LTRE)}\label{year-to-year-differences-in-population-growth-rate-fixed-design-ltre}}

TBA

\hypertarget{quantifying-demographic-contributions-to-long-term-population-trends}{%
\section{Quantifying demographic contributions to long-term population trends}\label{quantifying-demographic-contributions-to-long-term-population-trends}}

TBA

\hypertarget{differences-in-population-trajectories-between-time-periods-period-design-ltre}{%
\subsection{Differences in population trajectories between time periods (period design LTRE)}\label{differences-in-population-trajectories-between-time-periods-period-design-ltre}}

TBA

\hypertarget{ExtOutlook}{%
\chapter{Useful extensions and outlook}\label{ExtOutlook}}

This last chapter is dedicated to possible future applications of \texttt{SPI-IPM}. For now, it is simply a collection of section titles alluding to potentially useful and interesting extensions of the model focusing on accommodating other species and additional data sources, as well as building further on the potential of multi-population studies.
The collection of section titles may serve as inspiration, but ideally, future users of \texttt{SPI-IPM} will contribute and describe their model extensions in this chapter.

\hypertarget{adapting-the-population-model-for-your-speciespopulation}{%
\section{Adapting the population model for your species/population}\label{adapting-the-population-model-for-your-speciespopulation}}

\hypertarget{accounting-for-multiple-broods-per-bird-per-year}{%
\subsection{Accounting for multiple broods per bird per year}\label{accounting-for-multiple-broods-per-bird-per-year}}

\hypertarget{altering-age-structure}{%
\subsection{Altering age structure}\label{altering-age-structure}}

\hypertarget{individual-heterogeneity-beyond-age-sex-traits-and-more}{%
\subsection{Individual heterogeneity beyond age: sex, traits, and more}\label{individual-heterogeneity-beyond-age-sex-traits-and-more}}

\hypertarget{including-additional-data-and-informative-priors}{%
\section{Including additional data and informative priors}\label{including-additional-data-and-informative-priors}}

\hypertarget{including-partially-observed-age-information}{%
\subsection{Including partially observed age information}\label{including-partially-observed-age-information}}

\hypertarget{making-the-most-of-auxiliary-knowledge-about-immigrantsdispersers}{%
\subsection{Making the most of auxiliary knowledge about immigrants/dispersers}\label{making-the-most-of-auxiliary-knowledge-about-immigrantsdispersers}}

\hypertarget{letting-published-values-help-with-estimation-when-data-is-sparse}{%
\subsection{Letting published values help with estimation when data is sparse}\label{letting-published-values-help-with-estimation-when-data-is-sparse}}

\hypertarget{building-on-the-multi-population-perspective}{%
\section{Building on the multi-population perspective}\label{building-on-the-multi-population-perspective}}

\hypertarget{joint-analysis-of-data-from-several-populations}{%
\subsection{Joint analysis of data from several populations}\label{joint-analysis-of-data-from-several-populations}}

\hypertarget{modelling-cross-population-covariation}{%
\subsection{Modelling cross-population covariation}\label{modelling-cross-population-covariation}}

\hypertarget{estimating-hyper-parameters-in-large-scale-analyses}{%
\subsection{Estimating hyper-parameters in large-scale analyses}\label{estimating-hyper-parameters-in-large-scale-analyses}}

\hypertarget{unlocking-the-secrets-of-dispersal}{%
\subsection{Unlocking the secrets of dispersal}\label{unlocking-the-secrets-of-dispersal}}

  \bibliography{book.bib,packages.bib}

\end{document}
